\documentclass[8]{report}
\usepackage[usenames]{color} %used for font color
\usepackage{amssymb} %maths
\usepackage{amsmath} %maths
\usepackage[utf8]{inputenc} %useful to type directly diacritic characters
\usepackage{booktabs}
\usepackage{subfig}
\usepackage{caption}
\usepackage{rotating}
\usepackage{adjustbox}
\usepackage{threeparttable}
\usepackage{array}

\def\sym#1{\ifmmode^{#1}\else\(^{#1}\)\fi}
\newcolumntype{C}{@{\extracolsep{1cm}}c@{\extracolsep{0pt}}}%


\begin{document}
\begin{table}[ht]



\begin{adjustbox}{width=1.35\textwidth}
\begin{threeparttable}

\begin{tabular}{l*{6}{c}}
 \multicolumn{4}{c}{Time Complexity:} \\
   &\multicolumn{1}{c}{}  &\multicolumn{1}{c}{}   &\multicolumn{1}{c}{} &\multicolumn{1}{c}{}  &\multicolumn{1}{c}{}   \\
\hline\hline


     \ File Number         &\multicolumn{1}{c}{ADT}  &\multicolumn{1}{c}{IndIns}   &\multicolumn{1}{c}{IndDel} &\multicolumn{1}{c}{SerIns}  &\multicolumn{1}{c}{SerDel}  &\multicolumn{1}{c}{Entire File}  \\


\hline


\ File 1 	&      1     &     O( n )  &    -   &   250 * O( n)    &  -   & n* O(n )   \\
                       

\ File 2 	&      1     &      O( n )   &   O( n ) 	&   125 * O( n )   &   125 * O( n )     & n * O( n )  \\
                  
\ File 3 	&      1     &      O( n )   &   O( n ) 	&    125 * O( n )   &    125 * O( n )     & n * O( n )  \\     
 \ File 4 	&      1     &      O( log n )   &   O( log n ) 	&    125 *O( log n )   &    125 * O( log n )     & n * O( log n )  \\

 

 
 
 

\ File 1 	&      2    &     O( log n)  &    -  		&   250 * O( log n )    &  -   &  n * O( log n )   \\
                       

\ File 2 	&      2     &      O( log n )   &   O( log n ) 	&    125 * O( log n )   &   125 *  O( log n )     & n* O( log n  )  \\
                  
\ File 3 	&      2     &      O( log n )   &   O( log n ) 	&    125 * O( log n )   &   125 *  O( log n )     & n* O( log n  )  \\ 
 \ File 4 	&      2     &      O( log n )   &   O( log n ) 	&    125 * O( log n )   &   125 *  O( log n )     & n* O( log n  )  \\ 
 
 

\ File 1 	&      3   &     O( log n)  &    -  		&   250 * O( log n )    &  -   &  n * O( log n )   \\
                       

\ File 2 	&      3     &      O( log n )   &   O( log n ) 	&    125 * O( log n )   &   125 *  O( log n )     & n* O( log n  )  \\
                  
\ File 3 	&      3     &      O( log n )   &   O( log n ) 	&    125 * O( log n )   &   125 *  O( log n )     & n* O( log n  )  \\ 
 \ File 4 	&      3     &      O( log n )   &   O( log n ) 	&    125 * O( log n )   &   125 *  O( log n )     & n* O( log n  )  \\ 

 
 
 
\ File 1 	&      4   &     O( log n)  &    -  		&   250 * O( log n )    &  -   &  n * O( log n )   \\
                       

\ File 2 	&      4     &      O( log n )   &   O( 1 ) 	&    125 * O( log n )   &   125 *  O( 1)     & n/2 * O(1 ) +  n/ 2 * O( log n/2  )  \\
                  
\ File 3 	&      4     &      O( log n )   &   O( 1 ) 	&    125 * O( log n )   &   125 *  O( 1)     & n/2 * O(1 ) +  n/ 2 * O( log n/2  )  \\
 \ File 4 	&      4     &      O( log n )   &   O( 1 ) 	&    125 * O( log n )   &   125 *  O( 1)     & n/2 * O(1 ) +  n/ 2 * O( log n/2  )  \\
 
 
 
 
\ File 1 	&      5    &     O( n )  &    -  		&   250 * O( n )    &  -   &  n * O( n )   \\
                       

\ File 2 	&      5     &      O(  n )   &   O( n ) 	&    125 * O(  n )   &   125 *  O( n )     & n* O(  n  )  \\
                  
\ File 3 	&      5     &      O(  n )   &   O(  n ) 	&    125 * O(  n )   &   125 *  O(  n )     & n* O( n  )  \\ 
 \ File 4 	&      5     &      O( n )   &   O( n ) 	&    125 * O( n )   &   125 *  O( n )     & n* O( n  )  \\ 
 
 
 
 
 

\ File 1 	&      6&     O( log n)  &    -  		&   250 * O( log n )    &  -   &  n * O( log n )   \\
                       

\ File 2 	&      6     &      O( log n )   &   O( log n ) 	&    125 * O( log n )   &   125 *  O( log n )     & n* O( log n  )  \\
                  
\ File 3 	&      6     &      O( log n )   &   O( log n ) 	&    125 * O( log n )   &   125 *  O( log n )     & n* O( log n  )  \\ 
 \ File 4 	&      6     &      O( log n )   &   O( log n ) 	&    125 * O( log n )   &   125 *  O( log n )     & n* O( log n  )  \\  
 
 
\ File 1 	&      7     &     O( ? + 1)   &   - &   250 * O( ? + 1)     &  -   &  n * O( ? + 1)   \\
                       

\ File 2 	&      7     &      O( ? + 1)    &  O( ? + 1)  	&    125 * O( ? + 1)    &    125 * O( ? + 1)     & n * O( ? + 1)   \\
                  
\ File 3 	&      7     &      O( ? + 1)    &  O( ? + 1)  	&    125 * O( ? + 1)    &    125 * O( ? + 1)     & n * O( ? + 1)   \\
 \ File 4 	&      7     &      O( ? + 1)    &  O( ? + 1)  	&    125 * O( ? + 1)    &    125 * O( ? + 1)     & n * O( ? + 1)   \\
 

\ File 1 	&      8  &     O(1/1-?)   &   - &   250 * O(1/1-?)     &  -   &  n * O(1/1-?)   \\
                       

\ File 2 	&      8     &      O(1/1-?)    &  O(1/1-?)  	&    125 * O(1/1-?)    &    125 * O(1/1-?)     & n * O(1/1-?)   \\
                  
\ File 3 	&      8     &      O(1/1-?)    &  O(1/1-?) 	&    125 * O(1/1-?)   &    125 * O(1/1-?)     & n * O(1/1-?)   \\
 \ File 4 	&      8     &      O(1/1-?)   &  O(1/1-?) 	&    125 * O(1/1-?)   &    125 * O(1/1-?)     & n * O(1/1-?)  \\
 
 
 
\ File 1 	&      9  &     O(1/1-?)   &   - &   250 * O(1/1-?)     &  -   &  n * O(1/1-?)   \\
                       

\ File 2 	&      9     &      O(1/1-?)    &  O(1/1-?)  	&    125 * O(1/1-?)    &    125 * O(1/1-?)     & n * O(1/1-?)   \\
                  
\ File 3 	&      9     &      O(1/1-?)    &  O(1/1-?) 	&    125 * O(1/1-?)   &    125 * O(1/1-?)     & n * O(1/1-?)   \\
 \ File 4 	&      9     &      O(1/1-?)   &  O(1/1-?) 	&    125 * O(1/1-?)   &    125 * O(1/1-?)     & n * O(1/1-?)  \\


                                                        


\hline 

\end{tabular}


\begin{tablenotes} 
\setlength\labelsep{0pt}
\footnotesize
\item \textit{Notes:} \\
 ? = Lambda \\
 IndIns = Individual insertion  \\			
IndDel = Individual deletion    \\			 
SerIns =  Entire series of insertions\\		
SerDel = Entire series of deletions\\		
	ADT 1 = Binary Search Tree\\		
	ADT 2 = AVL Tree\\
	ADT 3 = Splay Tree\\
	ADT 4 = Binary Heap\\					
	ADT  5 = Skip List\\							
ADT 6 = B-Tree \\
ADT 7 =  Separate Chaining Hash Table \\
ADT 8 =  Quadratic Probing Hash Table \\
ADT 9 = Quadratic Pointer Probing Hash Table\\


\end{tablenotes}
\end{threeparttable}
\end{adjustbox}
\end{table}
\end{document} 

























